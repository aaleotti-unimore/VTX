\chapter{Implementazione}
\label{implementazione}
In questa sezione è descritta l'implementazione del progetto definito nel capitolo precedente. I principali strumenti utilizzati sono Python come linguaggio di programmazione e le librerie Scikit-learn \cite{sklearn} per attuare machine learning e Keras \cite{keras} per realizzare reti neurali.
Di seguito sono elencati i dettagli implementativi delle singole entità, ed i parametri usati durante la fase sperimentale.

\section{Classificatore Random Forest}
\label{imp:randomforest}
Il classificatore Random Forest è stato implementato tramite l'uso della libreria python Scikit-learn \cite{sklearn}. 

\subsection{Dataset}
Il dataset utilizzato per le fasi di training e testing del classificatore è stato definito nella sezione \ref{pro:randomforestdataset}

\subsection{Parametri}
In particolare i parametri del classificatore utilizzato che sono stati modificati dal loro valore di default sono:
\begin{itemize}
\item \textbf{\texttt{n\_estimators}} = 100. Numero di stimatori da utilizzare nel processo.
\item \textbf{\texttt{min\_samples\_leaf}} = 50. Numero minimo di campioni per un nodo foglia.
\item \textbf{\texttt{oob\_score}} = True. Implica l'uso di campioni \textit{out-of-bag} per ottenere una stima facilitata dell'errore. Con questa tecnica una parte del campione di stima viene esclusa per essere utilizzata come insieme di verifica.
\end{itemize}
\section{Classificatore Neurale}
\label{imp:classneurale}

\section{Autoencoder}
\label{imp:autoencoder}

\section{Generative Adversarial Network}
\label{imp:gan}

\todo{parlare del pretraining}