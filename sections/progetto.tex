\chapter{Progetto}
\label{progetto}

In questo capitolo si propone il progetto realizzato per raggiungere gli obiettivi preposti: si è partiti dalla realizzazione di un classificatore basato su \textit{Random Forest} per poi passare ad una versione più elaborata, utilizzando una rete neurale. Il passo successivo ha riguardato la creazione di una \textit{Generative Adversarial Network} a partire da un Autoencoder.  


\section{Classificatore Random Forest}
\label{randomforest}
La prima fase di questo studio è stata quella di implementare un classificatore in grado di separare efficacemente domini DGA da domini non malevoli. 

\subsection{Input}
\label{randomforestinput}
I dataset di training e testing sono stati ricavati da \textit{Alexa Top 1M} per quel che riguarda i domini non malevoli, mentre grazie al repository fornito da \cite{github:dgarepo} è stato possibile ottenere un \textit{dataset} esaustivo di esempi DGA da diverse famiglie

\subsection{Composizione Interna}
\label{randomforestinterno}

\subsection{Output}
\label{randomforestoutput}

\section{Classificatore Neurale}
\label{classificatorenn}

\subsection{Input}
\label{classificatorenninput}

\subsection{Composizione Interna}
\label{classificatorenninterno}

\subsection{Output}
\label{classificatorennoutput}

\section{Realizzazione Adversarial Learning}
\label{adv}

\subsection{Input}
\label{advinput}

\subsection{Composizione Interna}
\label{advinterno}

\subsection{Output}
\label{advoutput}

