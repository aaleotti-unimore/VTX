\chapter{Progetto}
\label{progetto}

In questo capitolo si propone il progetto realizzato per raggiungere gli obiettivi preposti: si è partiti dalla realizzazione di un classificatore basato su Random Forest~\ref{randomforest} per poi passare ad una versione più elaborata, utilizzando una rete neurale. Il passo successivo ha riguardato la creazione di una Generative Adversarial Network a partire da un Autoencoder.  

la Sezione~\ref{citazioni} descrive come scrivere citazioni, la Sezione~\ref{oggetti-float} propone degli esempi di oggetti float, la Sezione~\ref{compilazione} descrive come compilare questo documento.


\section{Classificatore Random Forest}
\label{randomforest}

\subsection{Input}
\label{randomforestinput}

\subsection{Composizione Interna}
\label{randomforestinterno}

\subsection{Output}
\label{randomforestoutput}

\section{Classificatore Neurale}
\label{classificatorenn}

\subsection{Input}
\label{classificatorenninput}

\subsection{Composizione Interna}
\label{classificatorenninterno}

\subsection{Output}
\label{classificatorennoutput}

\section{Realizzazione Adversarial Learning}
\label{adv}

\subsection{Input}
\label{advinput}

\subsection{Composizione Interna}
\label{advinterno}

\subsection{Output}
\label{advoutput}

%\section{Citazioni}
%\label{citazioni}
%
%Inserisco qualche citazione per mostrare la bibliografia. Per gli articoli accademici è quasi sempre possibile reperire i blocchi da inserire nel file bib da scholar~\cite{google:scholar}, come ad esempio~\cite{feige:zero}. Scholar in questo caso è una risorsa/sito online e per questo. Precediamo le citazione da uno spazio indivisibile tramite il carattere \textasciitilde.
%
%\section{Oggetti float}
%\label{oggetti-float}
%
%Nella Sezione~\ref{figure-float} si propone un esempio di figura float, mentre nella Sezione~\ref{tabelle-float} si propone un esempio di tabella float.
%
%\subsection{Figure}
%\label{figure-float}
%
%La Figura~\ref{fig:esempio} è un esempio di figura float.
%
%\begin{figure}[htb]
%    \centering
%    \includegraphics[width=.4\columnwidth]{figures/example.pdf}
%    \caption{Esempio di figura float in latex.}
%\label{fig:esempio}
%\end{figure}
%
%\subsection{Tabelle}
%\label{tabelle-float}
%
%La Tabella~\ref{tab:esempio} è un esempio di tabella.
%
%\begin{table}[htb]
%    \centering
%    \begin{tabular}{| c | l | r |}
%        \hline
%        allineamento centrale & allineamento a sinistra  & allineamento a destra
%        \\
%        \hline
%        \hline
%        centrale & sinistra & destra
%        \\
%        \hline
%    \end{tabular}
%    \caption{Esempio di tabella float in latex.}
%\label{tab:esempio}
%\end{table}
%
%\section{Compilazione}
%\label{compilazione}
%
%Di seguito il codice da utilizzare per generare il pdf:
%\begin{lstlisting}[language=bash]
%$ pdflatex main.tex
%$ bibtex main.aux
%$ pdflatex main.tex
%$ pdflatex main.tex
%\end{lstlisting}

